\documentclass{article}
\usepackage[utf8]{inputenc}
\usepackage{amsmath}

\setlength{\topmargin}{0.1in}
\setlength{\oddsidemargin}{0in}
\setlength{\evensidemargin}{0in}
\setlength{\headheight}{0in}
\setlength{\headsep}{0in}
\setlength{\textheight}{9in}
\setlength{\textwidth}{6.5in}
\title{CS515 HW1}
\author{Steve Hill}
\date{October 2015}
\usepackage{mathtools}
\DeclarePairedDelimiter{\ceil}{\lceil}{\rceil}
\DeclarePairedDelimiter{\floor}{\lfloor}{\rfloor}


\begin{document}

\maketitle

\begin{enumerate}
\item 
    \begin{enumerate}
    \item 
    Prove that the following algorithm actually sorts its input.
    %Solution goes here%

    \item
    Would StoogeSort still sort correctly if we replaced $m = \ceil{2n/3e}$ with $m = \floor{2n/3c}$? Justify
your answer.
    %Solution goes here%
 
    \item
    State a recurrence (including the base case(s)) for the number of comparisons executed by
StoogeSort.
    %Solution goes here%

    \item 
    Solve the recurrence, and prove that your solution is correct. 
    %Solution goes here%

    \item
    Prove that the number of swaps executed by StoogeSort is at most $n \choose 2$
    %Solution goes here%

    \end{enumerate}

\item 
An inversion in an array A[1..n] is a pair of indices $(i, j)$ such that $i < j$ and
$A[i] > A[j]$. The number of inversions in an n-element array is between 0 (if the array is sorted)
and $n \choose 2$ (if the array is sorted backward). Describe an algorithm to count the number of inversions
in an n-element array in $O(n log n)$ time. Prove your algorithm is correct and analyze its running
time

%Solution goes here%

\item 
A shuffle of two strings X and Y is formed by interspersing the characters into a new string,
keeping the characters of X and Y in the same order. For example, the string BANANAANANAS
is a shuffle of the strings BANANA and ANANAS in several different ways.
Similarly, the strings PRODGYRNAMAMMIINCG and DYPRONGARMAMMICING are both
shuffles of DYNAMIC and PROGRAMMING:
Given three strings $A[1..m], B[1..n]$, and $C[1..m + n]$, describe an algorithm to determine
whether C is a shuffle of X and Y. Prove your algorithm is correct and analyze its running time.
%Solution goes here%

\[ isShuffle(C, X, Y) 
    \begin{cases}
    
    True & \text{if } RecursiveHelper(C,X,Y) = length(C) \\
    False, & \text{otherwise}
    \end{cases}
\]

Consider the following recurrence relation:


$\left RecursiveHelper(C, X, Y)$:\\
\[
    max  
    \begin{cases}
        RecursiveHelper(C[1:], X[1:], Y) + 1   & \text{if } C[0] = X[0])\\
        RecursiveHelper(C[1:], X, Y[1:]) + 1    & \text{if } C[0] = Y[0]\\
        1  & \text{if } C(i) = X(i) \text{ and } length(Y) = 0\\
        1  & \text{if } C(i) = Y(i) \text{ and } length(X) = 0\\
        0, & \text{otherwise}
    \end{cases}
\]
The base case is defined as if C only contains one element and there is a match, return 1. If there is no match, return 0. This allows isShuffle to evaluate the length at the end of the recurrence as a measure of if C is a shuffle of X and Y.
The recurrence is looking for a match from either the first position in X or Y to C, if there is a match, it removes the beginning of C and the corresponding character in X or Y and recurses. \\

Proof of correctness:\\
\\
$Theorem:$ The above recurrence relation solves the shuffled string problem. \\
$Proof:$ \\
Let C be length 1, X be length 0, and Y be length 0. RecursiveHelper would return 0 by the last case and is Shuffle would return false. \\
Let C be Length 1, X be length 1, and Y be length 0. If X is the same character as C, it will return 1, which is the length of the string, thus isShuffle returns true and the algorithm is correct. Similariy, if they are not equal, then the algorithm will reach the last case and return false\\
Let k be the length of C. Let i,j be the length of X, Y respectively. \\
Let $k > 1$ and length of X + length of $Y > 1$: \\
\\
For k = 2, it is easy to see that both X and Y must be one character and contained in C in order for isShuffle to return True, if there are two matches, case 3 and case 4 will respectively be met in both paths. Thus the max will be 2 and the algorithm will succeed and correctly return true. If there is no match, both RecursiveHelper calls will return 1 and isShuffle will return true. From here it is easy to see that for k+1, isShuffle will always return the correct answer.
\\
QED\\

$Dynamic Programming:$\\
\\
isShuffleDP(C, X, Y)\\

Construct a $n+1 x m+1$ table denoted T. At position 0x0, place a 0. \\
Move through the table from left to right and suppose i and j represent the current indicies\\
In each cell, enter the following: \\
    \[
     max
    \begin{cases}
        T(i - 1, j) + 1  & \text{if } C[i+j] = X[i])\\
        T(i, j - 1) + 1  & \text{if } C[ni+j] = Y[j])\\
        0, & \text{otherwise}
    \end{cases}
    \]
This will populate every cell with the computations required the isShuffled recurrence relation in $O(mn)$ time using $O(mn)$ space. Either isShuffled may be run using the table as supplementary table, or you may use the cell at position n,m. If it is equal to the length of C, then the string is shuffled, otherwise C is not a shuffle of X and Y.


\item 
You are driving a bus along a highway, full of rowdy, hyper, thirsty students
and a soda fountain machine. Each minute that a student is on your bus, that student drinks one
ounce of soda. Your goal is to drop the students off quickly, so that the total amount of soda
consumed by all students is as small as possible.

You know how many students will get off of the bus at each exit. Your bus begins somewhere
along the highway (probably not at either end) and move s at a constant speed of 3.14 miles per
hour. You must drive the bus along the highway; however, you may drive forward to one exit then
backward to an exit in the opposite direction, switching as often as you like. (You can stop the
bus, drop off students, and turn around instantaneously.)

Describe an efficient algorithm to drop the students off so that they drink as little soda as
possible. Your input consists of the bus route (a list of the exits, together with the travel time
between successive exits), the number of students you will drop off at each exit, and the current
location of your bus (which you may assume is an exit). Prove your algorithm is correct and analyze
its running time.

%Solution goes here%

\end{enumerate}

\end{document}

